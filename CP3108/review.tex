\PassOptionsToPackage{unicode=true}{hyperref} % options for packages loaded elsewhere
\PassOptionsToPackage{hyphens}{url}
\PassOptionsToPackage{dvipsnames,svgnames*,x11names*}{xcolor}
%
\documentclass[british,a4paper,]{scrartcl}
\usepackage[british]{babel}
\title{CP3208 Literature Review -- Automaticity of Algebraic Structures}
\author{Qi Ji}
\date{March 2019}
\makeatletter
\let\thetitle\@title
\let\theauthor\@author
\let\thedate\@date
\makeatother

\usepackage{lmodern}
\usepackage{amssymb,amsmath,amsthm}
\usepackage{ifxetex,ifluatex}
\ifnum 0\ifxetex 1\fi\ifluatex 1\fi=0 % if pdftex
  \usepackage[T1]{fontenc}
  \usepackage[utf8]{inputenc}
  \usepackage{textcomp} % provides euro and other symbols
\else % if luatex or xelatex
  \usepackage{unicode-math}
  \defaultfontfeatures{Ligatures=TeX,Scale=MatchLowercase}
\fi
\usepackage{fancyhdr}
\usepackage{tabu}
\usepackage[tablewithin=section]{caption}
\usepackage{xcolor}
\usepackage{hyperref}
\hypersetup{
            pdftitle={\thetitle},
            pdfauthor={\theauthor},
            colorlinks=true,
            linkcolor=Violet,
            citecolor=PineGreen,
            urlcolor=MidnightBlue,
            breaklinks=true}
\usepackage{cleveref}

% use upquote if available, for straight quotes in verbatim environments
\IfFileExists{upquote.sty}{\usepackage{upquote}}{}
% use microtype if available
\IfFileExists{microtype.sty}{%
\usepackage[]{microtype}
\UseMicrotypeSet[protrusion]{basicmath} % disable protrusion for tt fonts
}{}

% set default figure placement to htbp
\makeatletter
\def\fps@figure{htbp}
\makeatother

\usepackage{csquotes}
%% load polyglossia as late as possible as it *could* call bidi if RTL lang (e.g. Hebrew or Arabic)
%\usepackage{polyglossia}
%\setmainlanguage[variant=british]{english}
\usepackage[
    style=alphabetic,
    maxalphanames=6,
    maxnames=6,
    doi=false,isbn=false,url=false,
]{biblatex}
\addbibresource{\jobname.bib}

\newtheorem{theorem}{Theorem}
\newtheorem{question}[theorem]{Question}
\theoremstyle{definition}
\newtheorem{definition}[theorem]{Definition}
\newtheorem{example}[theorem]{Example}
\theoremstyle{remark}
\newtheorem*{remark}{Remark}

\newcommand{\set}[1]{\left\{\, #1 \,\right\}}
\newcommand{\abs}[1]{\left\lvert #1 \right\rvert}
\newcommand{\sq}[1]{\left[ #1 \right]}
\newcommand{\N}{\mathbb{N}}
\newcommand{\Q}{\mathbb{Q}}
\newcommand{\Z}{\mathbb{Z}}

\pagestyle{fancy}
\lhead{CP3208 Literature Review}
\chead{}
\rhead{\thedate}
\lfoot{\theauthor}
\cfoot{}
\rfoot{\thepage}
\renewcommand{\footrulewidth}{\headrulewidth}

\begin{document}
\maketitle

There are numerous connections between group theory and automata theory.
Algebraic structures like groups could be described using automata theory, while
concepts from algebra also come in useful in automata theory.

\section{Semigroups, Monoids and Groups}

We will first introduce some relevant concepts regarding common algebraic structures.

Let \(G\) be a set and let \(\circ : G\times G \to G\).
\begin{definition}
    The structure \((G,\circ)\) is called a \textbf{semigroup} iff \(\circ\) is associative,
    that is for all \(x,y,z,\in G\), \(x\circ(y\circ z) = (x\circ y)\circ z\).
\end{definition}
\begin{definition}
    The structure \((G, \circ, e)\) is called a \textbf{monoid} iff \((G, \circ)\) is a semigroup
    and for all \(x\in G\), \(x\circ e = e \circ x = x\).
\end{definition}
\begin{definition}
    The structure \((G, \circ, e)\) is called a \textbf{group} iff \((G, \circ, e)\) is a monoid
    and for all \(x\in G\), there exists \(y \in G\) satisfying \(x\circ y = e\).
\end{definition}
\begin{example} \label{example:integers}
    The structure \((\set{1,2,\dots}, +)\) is a semigroup.
    Adding \(0\) to it gives us the monoid of natural numbers \((\N, +, 0)\).
    Considering its closure gives us the additive group of integers \((\Z, +, 0)\).
\end{example}
\begin{definition} \label{def:finitelygenerated}
    A semigroup \((G, \circ)\) is called \textbf{finitely-generated} iff
    there exists finite subset \(F \subseteq G\) where
    for each \(x \in G\), there exists \(y_1, y_2,\dots, y_n\in A\) satisfying
    \(x = y_1\circ y_2 \circ \dots \circ y_n\).
    The set \(F\) is set of generators for \(G\).
\end{definition}

\begin{remark}
    \Cref{def:finitelygenerated} generalises to monoids and groups in the obvious way.
    Also note that the structures in \Cref{example:integers} are finitely-generated,
    with the generator \(1\).
\end{remark}

Groups have been studied by mathematicians for centuries in their own right,
but only recently have the connection between group theory and automata theory been made.\autocite{Ho76,Ho83,KN,Epstein}

\section{Notions of Automaticity for Groups and Monoids}

The low complexity of automata motivates the search for automatic presentations of various algebraic structures.
For example,
the word problem for groups in general is well-known to be undecidable.
However, most reasonable formalisations of automatic groups will have their word problem solvable with a
quadratic time complexity.\autocite{Epstein,KKM}

We first introduce the framework proposed by Hodgson\autocite{Ho76, Ho83} and
Khoussainov and Nerode\autocite{KN}.
\begin{definition} \label{defn:fullyauto}
     A semigroup \((G,\circ)\) is called \textbf{fully automatic} iff
     \begin{itemize}
         \item \(G\) is regular over \(\Sigma^*\) with \(\Sigma\) being a finite alphabet,
         \item \(\circ: G\times G \to G\) is an automatic function.
     \end{itemize}
     Fully automatic monoids and groups are defined analogously.
\end{definition}
\begin{remark}
    In the original literature the authors referred to their definition as just `automatic''.
    We follow the convention in \autocite{fullautomatatheory} and use the term ``fully automatic''
    in the sense that the \emph{full} semigroup operation is automatic.
    At the same time, we also disambiguate it from the following more popular definition,
    attributed to Epstein, Cannon, Holt, Levy, Paterson and Thurston\autocite{Epstein}.
\end{remark}

\begin{definition} \label{defn:epsteinauto}
    Let \((G, \circ)\) be a semigroup generated by a finite subset \(F \subseteq G\).
    \((G, \circ)\) is \textbf{automatic} iff
    \begin{itemize}
        \item \(G\) is a regular subset of \(F^*\),
        \item each \(x\in G\) has exactly a unique representative in \(F^*\), and
        \item for each \(y \in G\), the multiplication map \((\circ y) : G \to G\) defined
            as \(x \mapsto x\circ y\) is automatic.
    \end{itemize}
    The notion of having exactly 1 representative is equivalent to allowing multiple representatives but
    with equality being automatic.
    To see why, consider how we can unambiguously choose the
    lexiographically smallest element as the unique representative.
\end{definition}

We can make some observations to
highlight the differences between \Cref{defn:fullyauto} and \Cref{defn:epsteinauto}.
\begin{theorem}
    There exists a semigroup which is automatic but not fully automatic.
\end{theorem}
\begin{proof}
    Theorem 12.22 in \autocite{fullautomatatheory} provides an explicit construction.
\end{proof}

Kharlampovich, Khoussainov and Miasnikov were the first to formally consider the related concept of a Cayley automatic group \autocite{KKM}.
It is sometimes also called graph automatic because it considers the Cayley graph of a group as the automatic structure.
\begin{definition} \label{defn:cayleyauto}
    A finitely generated group \(G\) generated by \(F\) is \textbf{Cayley automatic} iff
    the following conditions hold for some finite alphabet \(\Sigma\),
    \begin{itemize}
        \item representatives of \(G\) is regular in \(\Sigma^*\),
        \item each \(x\in G\) has a unique representative in \(\Sigma^*\),
        \item for each \(y\in F\), the right multiplication by \(y\) map is automatic.
    \end{itemize}
\end{definition}

\begin{remark}
    We note that \Cref{defn:epsteinauto} is a special case of this,
    with the extra condition that natural representatives are chosen,
    that is \(\Sigma = F\) the generating set.
    As a result \Cref{defn:cayleyauto} defines a strictly bigger class of automatic groups.
\end{remark}

\begin{example}
    The Heisenberg group \(\mathcal{H}_3(\Z)\) defined as
    \[\set{\begin{pmatrix}1&a&b\\0&1&c\\0&0&1\end{pmatrix}: a,b,c\in\Z}\]
    is Cayley automatic (6.6 in \autocite{KKM}), but not automatic (8.1.1 in \autocite{Epstein}).
\end{example}

\begin{definition} \label{defn:cayleybiauto}
    \(G\) is \textbf{Cayley biautomatic} if
    the third condition in \Cref{defn:cayleyauto} also holds for left multiplication.
\end{definition}

It is interesting to note that there exists a Cayley automatic group that is not Cayley biautomatic.
\autocite{MS}

\section{General Automatic Structures}

For general structures we consider the more general framework of
Hodgson\autocite{Ho76, Ho83} and Khoussainov and Nerode\autocite{KN}.

\begin{definition} \label{defn:automaticstructure}
    A structure \((A, R_1, R_2, \dots, R_k, f_1, f_2,\dots, f_h)\) is automatic iff
    \begin{itemize}
        \item the set \(A\) is regular in \(\Sigma^*\) where \(\Sigma\) is some finite alphabet,
        \item all the relations \(R_1, R_2, \dots, R_k\) are automatic, and
        \item all the functions \(f_1, f_2, \dots, f_h\) are automatic.
    \end{itemize}
    We do not distinguish between structures that are automatic and structures
    that are merely isomorphic to an automatic structure.
\end{definition}

\begin{example}
    A fully automatic group \((G,\circ)\) satisfying \Cref{defn:fullyauto} is an automatic structure.
\end{example}

\begin{definition}[Ordinals]
    A set \(\alpha\) is an ordinal iff every \(\beta\in\alpha\) is a subset of \(\alpha\) and
    \((\alpha, \in)\) is a well-order.
\end{definition}
Ordinals can be thought of as equivalence classes of well-ordered sets in set theory.
They naturally describe how many times a process is iterated, possibly transfinitely many times, and are commonly encountered in model theory and recursion Theory.
It is hence expected for there to be a suitable characterisation of ordinals in automata theory.\autocite{delhomme}
\begin{theorem}[Delhomm\'e] \label{thm:delhomme}
    Let \(\alpha\) be an ordinal, \((\alpha, +, \in)\) is automatic iff \(\alpha < \omega^\omega\).
\end{theorem}
\begin{proof}
    Theorem 13.10 in \autocite{fullautomatatheory} provides a proof in English.
\end{proof}

\section{Semiautomatic Structures and Groups}

Seeking more general ways to utilise finite automata for representing non-automatic structures,
Jain, Khoussainov, Stephan, Teng and Zou
proposed semiautomatic structures as a generalisation of \Cref{defn:automaticstructure}\autocite{semiauto}.
\begin{definition}
    Let \(f: R^n \to R\) be a function.
    \(f\) is \textbf{semiautomatic} iff fixing \(n-1\) inputs, the resultant \(R\to R\) function is automatic.
    The definition of a semiautomatic relation is analogous since we can view it as a \(\set{0,1}\)-valued function.
\end{definition}

\begin{definition}
    A structure \((A, f_1, f_2, \dots, f_k; g_1, g_2, g_h)\) is \textbf{semiautomatic} iff
    \begin{itemize}
        \item the set \(A\) is regular in \(\Sigma^*\) where \(\Sigma\) is some finite alphabet,
        \item all the functions \(f_1, f_2, \dots, f_k\) are automatic, and
        \item all the functions \(g_1, g_2, \dots, g_h\) are semiautomatic
    \end{itemize}
    where without loss of generality we treat relations as functions.
\end{definition}

Jain, Khoussainov and Stephan used semiautomaticity to describe group-like structures.\autocite{finitelysanjay}
They cited a related result by Tsankov.\autocite{tsankov}
\begin{theorem}
    The structure \((\Q, +, =)\) has no automatic presentation.
\end{theorem}
\begin{theorem}[21 in \autocite{semiauto}]
    The ordered group \((\Q,<,=;+)\) of rationals is semiautomatic.
\end{theorem}

We note that \(\Q\) is not finitely generated, which leads on to this still open question.
\begin{question}
    Are all finitely generated semiautomatic groups Cayley automatic?
\end{question}

Several important results presented in \autocite{semiauto} concern the automatic and semiautomatic presentations
of naturally-occuring algebraic structures.
\begin{theorem} \label{sqrtring}
    The ordered rings
    \begin{itemize}
        \item \((\Z[\phi], +, <, =; \cdot)\) where \(\phi = \frac{1+\sqrt{5}}2\) is the golden ratio, and
        \item \((\Z(\sqrt{n}), \Z, +, <, =; \cdot)\) for every natural number \(n\)
    \end{itemize}
    admit semiautomatic presentations.
\end{theorem}

\section{Future Directions}

We can possibly gain more insights in the automaticity of various algebraic structures.
A good place to start would be to consider a natural generalisation of \Cref{sqrtring}
\begin{question}
    Let \(n\) be a natural number.
    Consider the ordered ring \((\Z(\sqrt[3]{n}), +, \cdot, <)\).
    What can be made semiautomatic? What can be made automatic?
\end{question}
This could be generalised further.
\begin{question}
    Let \(n,p\) be a natural numbers.
    What about \((\Z(\sqrt[p]{n}), +, \cdot, <)\)?
\end{question}

We could also explore the structures of ordinals and
consider some generalisations to \Cref{thm:delhomme}.
\begin{question}
    Let \(\alpha \geq \omega^\omega\) be an ordinal.
    Consider the structure \((\alpha, \in, +, \cdot)\).
    What relations admit semiautomatic representations?
\end{question}

\printbibliography
\end{document}
