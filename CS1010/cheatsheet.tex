\documentclass[11pt,twocolumn]{scrartcl}
\usepackage[utf8]{inputenc}
\usepackage[british]{babel}
\usepackage[a4paper,landscape,top=1.2cm,bottom=1.2cm,left=1.2cm,right=1.2cm]{geometry}
\usepackage{amsmath,amssymb}
\usepackage{multirow}

\newcommand{\ul}[1]{\underline{\texttt{#1}}}

\usepackage{listings}
\usepackage[lighttt]{lmodern}

\makeatletter
\let\commentfullflexible\lst@column@fullflexible
\makeatother
\lstset{
    language=C,
    basicstyle=\ttfamily,
    commentstyle=\rmfamily\commentfullflexible,
%    keywordstyle=\bfseries,
%    stringstyle=\ttfamily,
    columns=fixed,
    showspaces=false,
    showstringspaces=false,
}

\lstMakeShortInline[columns=fixed]@
\setlength\parindent{0pt}
\begin{document}
\pagestyle{empty}
\paragraph{Operator Precedence Table}\hfill\\

\begin{tabular}{| l | l | l |}
    \hline
    Operator        & Desc.         & Direction   \\ \hline
    \verb|i++ i--|  & post-in/decr  & \multirow{ 2}{*}{$\longrightarrow$}   \\
    \verb|() [] . ->| & fn, arr, struct, struct thru pointer &          \\
    \hline
    \verb|++i --i|  & pre-in/decr & \multirow{ 5}{*}{$\longleftarrow$}    \\
    \verb|+ -|      & unary $+,-$ & \\
    \verb|! ~|      & $\lnot$, bitwise NOT & \\
    \verb|(type)|   & cast & \\
    \verb|* &|      & deref, addr of & \\
    \hline
    \verb|* / %|    & $\times$,$\div$, mod  & \multirow{7}{*}{$\longrightarrow$}  \\\cline{1-2}
    \verb|+ -|      & binary $+,-$          &   \\\cline{1-2}
    \verb|< <= > >=|& relational $<,\leqslant,>,\geqslant$ &   \\\cline{1-2}
    \verb|== !=|    & relational $=, \ne$   &   \\\cline{1-2}
    \verb@& ^ |@    & Bitwise AND,XOR,OR    &   \\\cline{1-2}
    \verb@&&@       & Logical $\land$       &   \\\cline{1-2}
    \verb@||@       & Logical $\lor$        &   \\
    \hline
    \verb|?:|       & Ternary               & \multirow{2}{*}{$\longleftarrow$}   \\\cline{1-2}
    \verb|= += -=| \dots & Assignment         & \\
    \hline
    \verb|,|        & Comma    & $\longrightarrow$  \\
    \hline
\end{tabular}

\paragraph{stdlib.h rand, time.h time} \hfill\\
\hfill\\
@int rand(void);@\\
The \verb|rand()| function computes a sequence of pseudo-random integers in the
range of \verb|0| to \verb|RAND_MAX|.

\hfill\\
@void srand(unsigned seed);@\\
The \verb|srand()| function sets its argument \ul{seed} as the seed for a new
sequence of pseudo-random numbers to be returned by \verb|rand()|.  These
sequences are repeatable by calling \verb|srand()| with the same seed value.

\hfill\\
@time_t time(time_t *tloc);@\\
The \verb|time()| function returns the value of time in seconds since 0 hours, 0 minutes,
0 seconds, January 1, 1970, Coordinated Universal Time, without including leap seconds.

\newpage
\paragraph{string.h}\hfill\\
\hfill\\
@char *strcpy(char *dst, char *src);@\\
The \verb|strcpy()| function copies the string \ul{src} to \ul{dst} (including
the terminating \verb|`\0'| character).

@char *strcat(char *s, char *append);@\\
The \verb|strcat()| function appends a copy of the null-terminated
string \ul{append} to the end of the null-terminated string \ul{s}, then add a
terminating \verb|`\0'|.  The string \ul{s} must have sufficient space to hold the
result.

\hfill\\
@char *strstr(char *big, char *little);@\\
The \verb|strstr()| function locates the first occurrence of the null-terminated
string \ul{little} in the null-terminated string \ul{big}.
If \ul{little} is an empty string, \ul{big} is returned; if \ul{little} occurs nowhere
in \ul{big}, \verb|NULL| is returned; otherwise a pointer to the first character of
the first occurrence of \ul{little} is returned.

@char *strchr(char *s, int c);@\\
The \verb|strchr()| function locates the first occurrence of \ul{c} (converted to a
\verb|char|) in the string pointed to by \ul{s}.  The terminating null character is
considered part of the string; therefore if \ul{c} is \verb|`\0'|, the functions
locate the terminating \verb|`\0'|.
Returns a pointer to the located character, or \verb|NULL| if the character does not
appear in the string.

\hfill\\
@char *strtok(char *str, char *sep);@\\
The \verb|strtok()| function is used to isolate sequential tokens in a null-
terminated string, \ul{str}.  These tokens are separated in the string by at
least one of the characters in \ul{sep}.  The first time that \verb|strtok()| is
called, \ul{str} should be specified; subsequent calls, wishing to obtain
further tokens from the same string, should pass a null pointer instead.
The separator string, \ul{sep}, must be supplied each time, and may change
between calls.
\begin{lstlisting}
char str[80] = "cs1010-abc!def@123";
char sep[10] = "-!@";
char *token;
token = strtok(str, sep);
while (token != NULL) {
    printf("%s\n", token);
    token = strtok(NULL, sep);
}
\end{lstlisting}

\newpage
\paragraph{File Processing}\hfill\\
@FILE *fopen(char *path, char *mode);@\\
The \verb|fopen()| function opens the file whose name is the string pointed to
by \ul{path} and associates a stream with it.

The argument \ul{mode} points to a string beginning with one of the following
letters:

\begin{tabular}{r p{0.9\linewidth}}
    \verb|"r"| &Open for reading.  The stream is positioned at the beginning of the file. \\
               &Fail if the file does not exist.  \\
    \verb|"w"| &Open for writing.  The stream is positioned at the beginning of the file. \\
               &Create the file if it does not exist.  \\
    \verb|"a"| &Open for writing.  The stream is positioned at the end of the file. \\
               &Create the file if it does not exist.
\end{tabular}
An optional \verb|"+"| following \verb|"r"|, \verb|"w"| or \verb|"a"| opens the file for both
reading and writing.

Upon successful completion \verb|fopen()| returns a \verb|FILE|
pointer.  Otherwise, \verb|NULL| is returned and the global variable \ul{errno} is
set to indicate the error.

@int feof(FILE *stream);@\\
@int ferror(FILE *stream);@\\
The function \verb|feof()| tests the end-of-file indicator for the stream
pointed to by \ul{stream}, returning non-zero if it is set.

The function \verb|ferror()| tests the error indicator for the stream pointed to
by \ul{stream}, returning non-zero if it is set.

\hfill\\
@int fscanf(FILE *stream, char *format, ...);@\\
Returns number of input items assigned, which can be fewer than provided for, or even zero,
in the event of a matching failure. Zero indicates that, while there was input available,
no conversions were assigned; typically this is due to an invalid input charactersuch as an alphabetic character for a `\verb|%d|' conversion.
The value \verb|EOF| is returned if an input failure occurs before any conversion
such as an end-of-file occurs.  If an error or end-of-file occurs after
conversion has begun, the number of conversions which were successfully
completed is returned.

\hfill\\
@char *fgets(char *str, int size, FILE *stream);@\\
The \verb|fgets()| function reads at most one less than the number of characters
specified by \ul{size} from the given \ul{stream} and stores them in the string
\ul{str}.  Reading stops when a newline character is found, at end-of-file or
error.  The newline, if any, is retained.  If any characters are read and
there is no error, a \verb|`\0'| character is appended to end the string.

@char *gets(char *str);@\\
The \verb|gets()| function is equivalent to \verb|fgets()| with an infinite \ul{size} and a
\ul{stream} of \verb|stdin|, except that the newline character (if any) is not stored
in the string. It is the caller's responsibility to ensure that the
input line, if any, is sufficiently short to fit in the string.

Upon successful completion, \verb|fgets()| and \verb|gets()| return a pointer to the
string.  If end-of-file occurs before any characters are read, they
return \verb|NULL| and the buffer contents remain unchanged.  If an error
occurs, they return \verb|NULL| and the buffer contents are indeterminate.  The
\verb|fgets()| and \verb|gets()| functions do not distinguish between end-of-file and
error, and callers must use \verb|feof(3)| and \verb|ferror(3)| to determine which
occurred.

\paragraph{File I/O Usage}\hfill\\
\begin{lstlisting}
FILE *readfp, *writefp;
...
if ((readfp = fopen("infile.in", "r")) == NULL) {
    printf("cannot open `infile.in' for reading\n");
    // bail/exit/return
}
if ((writefp = fopen("outfile.out", "w")) == NULL) {
    printf("cannot open `outfile.out' for writing\n");
    // bail/exit/return
}
/* do I/O on readfp/writefp for example */
char line[LENGTH+1];
while (fgets(line, LENGTH+1, readfp) != NULL) {
    // do stuff
}
if (!feof(readfp))
    // read error: bail/exit/return
fclose(readfp);
...
\end{lstlisting}

\paragraph{Corner cases}\hfill\\
\begin{lstlisting}
/* arrays */
char *str = "string literal";  // READ-ONLY MEMORY!
char str[20] = "modifiable";
int x[8] = {1,2,3};     // missing elements are all zeroes
int y[3] = {1,2,3,4};   // 4 will be truncated

/* 2nd dimension compulsory for matrix */
int m[4][3] = {         // 4x3 matrix
    { 1 },              // row 0 is { 1, 0, 0 }
    { 0, 1 },           // row 1 is { 0, 1, 0 }
};                      // everything else are zeroes
int m[3][2] = {1,3,2};  // will automatically be flattened.

int arr[2][3], arr2[3] = {1,1,1}
arr[0] = arr2;  // error: array type int[3] is not assignable

/* Struct */
typedef struct s_t {
    int a;
    float b;
    char* name;
} s_t;
s_t st = { 5, 2.2, "George" };
f(st);                  // structures are copied

/* I/O */
puts("blabla");         // write(1, "blabla\n", 7);
fputs("blabla", fp);    // write(?, "blabla", 6);
\end{lstlisting}

\end{document}
