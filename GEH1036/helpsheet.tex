\documentclass[11pt,twocolumn]{scrartcl}
\usepackage[utf8]{inputenc}
\usepackage[british]{babel}
\usepackage[a4paper,landscape,top=1cm,bottom=1cm,left=1cm,right=1cm]{geometry}

\usepackage{amsmath}
\usepackage{amssymb}
\usepackage{enumitem}

\usepackage{tikz,wrapfig}
\usetikzlibrary{shapes}

\usepackage{titlesec}
\titlespacing*{\subsection}
{0pt}{2ex}{2ex}
\titlespacing*{\paragraph}
{0pt}{1ex}{1ex}

\setlength\parindent{0pt}
\setitemize{noitemsep,topsep=0pt,parsep=0pt,partopsep=0pt}
\setenumerate{noitemsep,topsep=0pt,parsep=0pt,partopsep=0pt}

\begin{document}
%\setlength{\abovedisplayskip}{6pt}
%\setlength{\belowdisplayskip}{6pt}
%\setlength{\abovedisplayshortskip}{6pt}
%\setlength{\belowdisplayshortskip}{6pt}

\subsection*{1 Counting}
\paragraph{Useful formulas.}
\begin{align*}
    a + ar + ar^2 + \dots + ar^n &= \frac{a(r^{n+1} - 1)}{r - 1} \\
    a + (a+d) + \dots + (a+nd) &= \frac{(n+1)(a+a+nd)}{2} 	\\
    ^nP_k &= \frac{n!}{(n-k)!}		\\
    \binom{n}{k} = {}^nC_k &= \frac{n!}{k!(n-k)!}	\\
    \left(a+b\right)^n &= \sum_{k=0}^n \binom{n}{k} a^{n-k}b^k
\end{align*}

\subsection*{2 Graphing}
\paragraph{Definitions.}\hfill\\
\textbf{Tree.} graph with no cycle,
    \textbf{Leaf.} vertex with degree 1. \\
\textbf{Weight of graph.} sum of weights of all its edges.  \\
\textbf{Spanning tree.} subgraph that is a tree and contains \emph{all} vertices of original graph.

\paragraph{Theorems.}\hfill\\
\textbf{Degree Theorem.}In any graph, sum of degrees $= 2\times$ no. edges.   \\
\textbf{Leaf Lemma.} Every tree with $\geq 2$ vertices has $\geq 2$ leaves (deg1 vertices).\\
\textbf{Tree Theorem.} Every tree with $n$ vertices has exactly $n-1$ edges.    \\
\textbf{Euler Walk Theorem I.} A connected graph contains a closed Euler walk iff every vertex has an even degree.\\
\underline{directed version.} iff for every vertex the no. arrows in = no. arrows out.   \\
\textbf{Euler Walk Theorem II.} A connected graph contains an open Euler walk iff
\begin{enumerate}[label=(\roman*)]
    \item start/end vertices have odd degree; and
    \item all other vertices have even degree.
\end{enumerate}
\underline{directed version.} iff
\begin{enumerate}[label=(\roman*)]
    \item for start vertex, no. arrows out exactly one more than no. arrows in;
    \item for end vertex, no. arrows in exactly one more than no. arrows out;
    \item for all other vertices no. arrows in = no. arrows out.
\end{enumerate}

\paragraph{Prim's Algorithm}
\begin{enumerate}
    \item Choose any vertex to initalise tree.
    \item Grow the tree by one edge: of all edges that connect to vertices not yet in tree, find one with minimum-weight and add it in tree.
    \item Repeat step 2 until tree spans.
\end{enumerate}

\paragraph{Finding Euler circuit.}
\begin{enumerate}
    \item Check all vertices are even.
    \item Construct any cycle.
    \item If there's still remaining unused edges, construct cycle with those and combine the cycles.
    \item Repeat step 3 until all edges used.
\end{enumerate}

\paragraph{Chinese Postman Problem (Choosing edges to repeat)}
\begin{enumerate}
    \item Enumerate all pairings of odd vertices.
    \item For each pair, find paths that join the vertices with minimum weight.
    \item Use the set of pairings which sum of weights is minimised.
    \item Walk the edges found above twice.
\end{enumerate}

\paragraph{Vertex colouring}
If graph $G$ contains a complete graph on $n$ vertices, then $\chi(G) \ge n$.\\
Upper bound. Arrange degrees in decreasing order, place integers $1,2,3,\dots$ below the degrees until you reach integer $k$ such that $k+1 > d_{k+1}$, then $\chi(G) \le k+1$.

\subsection*{3 Clocking}
\paragraph{Congruence properties} If $a\equiv b\pmod{m}$, then I can
\begin{itemize}
    \item add multiples of modulo $m$ to any side
    \item multiply by integers
    \item exponentiate both sides with positive powers only
    \item two congruences with the same modulo can be added/multiplied with each other
\end{itemize}
\paragraph{Calendrical Knowledge} \hfill\\
YYYY is a leap year iff
\begin{itemize}
    \item it is not a century year and divisible by 4 (1996); or
    \item it is a century year and divisible by 400 (2000).
\end{itemize}
Calendar (mod 7)\\
\begin{tabular}{| c | c | c | c | c | c | c | c | c | c | c | c |}
    \hline
    Jan & Feb & Mar & Apr & May & Jun & Jul & Aug & Sep & Oct & Nov & Dec\\
    31 & 28/29& 31  & 30  & 31  & 30  & 31  & 31  & 30  & 31  & 30  & 31 \\
    %\hline
    3  & 0/1  & 3   & 2   & 3   & 2   & 3   & 3   & 2   & 3   & 2   & 3 \\
    \hline
\end{tabular}

\newpage
\subsection*{4 Coding}
\paragraph{Binary representation.} Divide by 2 and write it \textbf{right-to-left}.

\begin{tabular}{| r | c | c || r | c |}
    \hline
    bin & oct & hex & bin & hex\\
    \hline
    0   & 0     & 0 &1000 & 8 \\
    1   & 1     & 1 &1001 & 9 \\
    10  & 2     & 2 &1010 & A \\
    11  & 3     & 3 &1011 & B \\
    100 & 4     & 4 &1100 & C \\
    101 & 5     & 5 &1101 & D \\
    110 & 6     & 6 &1110 & E \\
    111 & 7     & 7 &1111 & F \\
    \hline
\end{tabular}

\paragraph{Modulo 37 encoding and ECC.} 0-9 becomes 0-9 and \verb|_|(space) is 36, and

\begin{tabular}{r || c c c c c c c c c c c c c}
    sym & A & B & C & D & E & F & G & H & I & J & K & L & M\\
    num & 10& 11& 12& 13& 14& 15& 16& 17& 18& 19& 20& 21& 22\\
    \hline\hline
    sym & N & O & P & Q & R & S & T & U & V & W & X & Y & Z\\
    num & 23& 24& 25& 26& 27& 28& 29& 30& 31& 32& 33& 34& 35\\
    \hline\hline
\end{tabular}
\begin{wrapfigure}{r}{0pt}
\def\firstcircle{(0,0) circle (1cm)}
\def\secondcircle{(0:1.2cm) circle (1cm)}
\def\thirdcircle{(300:1.2cm) circle (1cm)}
\begin{tikzpicture}
    \begin{scope}[shift={(3cm,-5cm)}]
%        \fill[red] \firstcircle;
%        \fill[green] \secondcircle;
%        \fill[blue] \thirdcircle;
        \draw \firstcircle  node[above,left]{$s_5$};
        \draw {(120:1.3cm)} node{$A$};
        \draw {(25:0.7cm)} node{$s_4$};
        \draw {(-25:0.7cm)} node{$s_1$};
        \draw {(280:0.7cm)} node{$s_3$};
        \draw \secondcircle node[above,right]{$s_6$};
        \draw {(0:1.2cm)+(60:1.3cm)} node{$B$};
        \draw {(0:1.2cm)+(270:0.7cm)} node{$s_2$};
        \draw \thirdcircle  node[below]{$s_7$};
        \draw {(300:1.2cm)+(-30:1.3cm)} node{$C$};
    \end{scope}
\end{tikzpicture}
\end{wrapfigure}

\paragraph{Weighted sum} $w = 1(\text{last char}) + 2(\text{second last char}) + \dots + n(\text{first char})$,
and for modulo 37 encoding, append a checksum char such that $w \equiv 0 \pmod{37}$.

\paragraph{ISBN.}
10 digits, X in last digit means 10, and a valid ISBN will have $w \equiv 0 \pmod{11}$.

\paragraph{Hamming $(7,4)$ and $(8,4)$ Codes.}

Every circle has even parity. In $(8,4), s_8$ keeps total parity even.

\hfill\\
\subsection*{5 Cryptography}
\paragraph{Affine Cryptosystems} with modulo $n$, $(a,b)$ is the encryption key\\
encrypt: $y \equiv ax + b \pmod{n}$\\
decrypt: $y \equiv a'x - a'b \pmod{n}$ where $a'$ is $a$ inverse modulo $n$.

\paragraph{Finding modulo Inverse.} Use Euclid $a = qd + r$, $\gcd(a,d) = \gcd(d,r)$.

\paragraph{Modular exponentiation.} Finding $a^n \pmod{m}$.
Express $n$ as multiples(usually binary), then split $a^n$ and find remainder for each term.

\newpage
\paragraph{RSA.}
$p,q$ are $2$ primes, $n = pq$, $k$ is an integer that has an inverse $\pmod{(p-1)(q-1)}$.
\begin{itemize}
    \item Let $P$ be a chunk of plaintext such that $0\leq P < n$.
    \item $j$ is inverse of $k$ mod $(p-1)(q-1)$, $kj \equiv 1 \pmod{(p-1)(q-1)}$
    \item Public key is $(n,k)$. Private key is $(j)$.
    \item Procedure encrypt: $C \equiv P^k \pmod{n}$.
    \item Procedure decrypt: $P \equiv C^j \pmod{n}$.
\end{itemize}

\subsection*{6 Chancing}
\paragraph{Probability.}
With event $E$ and sample space $S$,
$\mathbb{P}(E) = \displaystyle\frac{|E|}{|S|}$.

\paragraph{Independence.} characterising property: $\mathbb{P}(A\cap B) = \mathbb{P}(A)\mathbb{P}(B)$ where $A,B$ are events.

\paragraph{Conditional Probability.} $\mathbb{P}(A\mid B) = \displaystyle\frac{\mathbb{P}(A\cap B)}{\mathbb{P}(B)}$.

\paragraph{Binomial distribution.}
For $n$ independent events each with probability $p$,
\[
	X\sim B(n,p) \iff \mathbb{P}(X=k) = \binom{n}{k}p^k(1-p)^{n-k}
	\text{ and } \mathbb{E}[X] = np
\]
\paragraph{Expectation.} $\mathbb{E}[X] = \displaystyle\sum_x x\cdot\mathbb{P}(X = x)$ iterated through all possible values for $x$.\\
Expectation is linear so suppose total winnings $X = X_1 + \dots + X_k$, then
$$\mathbb{E}[X] = \mathbb{E}[X_1] + \dots + \mathbb{E}[X_k].$$

\paragraph{Poisson Distribution} For Binomial distribution with large $N$ and moderate $\mathbb{E}[X] = c = Np$, then
$$\mathbb{P}(X = k) \approx \frac{e^{-c}c^k}{k!}$$

%\subsection*{Misc}
\end{document}
