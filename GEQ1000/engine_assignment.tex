\PassOptionsToPackage{unicode=true}{hyperref} % options for packages loaded elsewhere
\PassOptionsToPackage{hyphens}{url}
\PassOptionsToPackage{dvipsnames,svgnames*,x11names*}{xcolor}
%
\documentclass[british,a4paper,]{article}
\usepackage[british]{babel}
\title{Cooling the YIH food court}
\author{Qi Ji (A0167793L) [D33]}
\date{26th October 2018}
\makeatletter
\let\thetitle\@title
\let\theauthor\@author
\let\thedate\@date
\makeatother

\usepackage{lmodern}
\usepackage{amssymb,amsmath,amsthm}
\usepackage{ifxetex,ifluatex}
\ifnum 0\ifxetex 1\fi\ifluatex 1\fi=0 % if pdftex
  \usepackage[T1]{fontenc}
  \usepackage[utf8]{inputenc}
  \usepackage{textcomp} % provides euro and other symbols
\else % if luatex or xelatex
  \usepackage{unicode-math}
  \defaultfontfeatures{Ligatures=TeX,Scale=MatchLowercase}
\fi
\usepackage{fancyhdr}
\usepackage{wasysym}
\usepackage{siunitx}
\usepackage{tabu}
\usepackage[tablewithin=section]{caption}

\usepackage{xcolor}
\usepackage{hyperref}
\hypersetup{
            pdftitle={GEQ1000 Engineering Assignment},
            pdfauthor={Qi Ji},
            colorlinks=true,
            linkcolor=Violet,
            citecolor=PineGreen,
            urlcolor=MidnightBlue,
            breaklinks=true}
\usepackage{cleveref}

% use upquote if available, for straight quotes in verbatim environments
\IfFileExists{upquote.sty}{\usepackage{upquote}}{}
% use microtype if available
\IfFileExists{microtype.sty}{%
\usepackage[]{microtype}
\UseMicrotypeSet[protrusion]{basicmath} % disable protrusion for tt fonts
}{}

% set default figure placement to htbp
\makeatletter
\def\fps@figure{htbp}
\makeatother

%\usepackage{csquotes}
%% load polyglossia as late as possible as it *could* call bidi if RTL lang (e.g. Hebrew or Arabic)
%\usepackage{polyglossia}
%\setmainlanguage[variant=british]{english}
\usepackage[style=numeric]{biblatex}
\addbibresource{\jobname.bib}

\theoremstyle{definition}
\newtheorem{solution}{Solution}
\newtheorem{consideration}{Consideration}
\newtheorem{definition}{Definition}[subsection]
\newtheorem{observation}[definition]{Observation}
%\theoremstyle{remark}

\newcommand{\set}[1]{\left\{\, #1 \,\right\}}

\pagestyle{fancy}
\lhead{GEQ1000}
\chead{}
\rhead{\thedate}
\lfoot{Qi Ji}
\cfoot{}
\rfoot{\thepage}
\renewcommand{\footrulewidth}{\headrulewidth}

\begin{document}
\maketitle

\begin{abstract}
    We propose a solution to the problem of high temperatures and humidity in the
    open-air food court area at Yusof Ishak House.
\end{abstract}

\tableofcontents

\section*{Introduction}

According the Meteorological Service Singapore, over the past decade,
there are 11.6 more warm days in every year, on average.\autocite{MSS}
In light of the numerous complaints regarding the open-air food court at Yusof Ishak House,
it is clear that some mistakes have been made in the design process of the food court.

\newpage

\section{Assessment}

\noindent\emph{This section addresses Question 1.}

\noindent After some preliminary research, we have narrowed down on four options.

\begin{solution} \label{sol1}
    Install 1$\times$ giant ceiling fan.
\end{solution}

\begin{solution} \label{sol2}
    Install 30$\times$ regular-sized ceiling fans, equidistant from each other.
\end{solution}

\begin{solution} \label{sol3}
    Install 40$\times$ regular standing fans, equidistant from each other.
\end{solution}

\begin{solution} \label{sol4}
    Install air-conditioning.
\end{solution}

%An immediate observation becomes clear.
%
%\begin{observation}
%    Solutions \ref{sol2} and \ref{sol3}, as stated, are physically impossible.
%\end{observation}
%\begin{proof}
%    Treat the ceiling or floor as a $2$-dimensional plane,
%    it follows from elementary geometry that there can be at most $3$ pairwise-equidistant points.
%\end{proof}
%
%\setcounter{solution}{1}
%\begin{solution}[fixed] \label{sol2fix}
%    Install 30$\times$ regular-sized ceiling fans, \emph{evenly-distributed}.
%\end{solution}
%\begin{solution}[fixed] \label{sol3fix}
%    Install 40$\times$ regular standing fans, \emph{evenly-distributed}.
%\end{solution}

\subsection{Priorities and considerations}

In selecting the best solution, we have settled on the top three most important considerations, listed in decreasing order of importance.

\begin{consideration} \label{consider:comfort}
    Comfort.

    We prioritise the comfort of our students and staff, as a more comfortable environment improves productivity and reduces stress.
\end{consideration}
\begin{consideration} \label{consider:cost}
    Cost efficiency.

    Rather than going with the cheapest option, we prioritise ``bang for the buck'',
    the solution selected should be worth what it costs.
\end{consideration}
\begin{consideration} \label{consider:aesthetic}
    Aesthetic appeal.

    The solution must look nice and fit in the architectural design of the existing ambience of Yusof Ishak House.
\end{consideration}

\subsection{Comparison of solutions}

For \Cref{consider:comfort}, we judge each solution based on the perceived temperature.
\begin{definition}
    The \textbf{perceived temperature} of a solution is defined as the actual temperature subtracted by
    expected improvements from airflow circulation, temperature control or humidity control.
\end{definition}

We assume here that the actual daily temperature is around \SI{30}{\celsius} on a warm day.
We also assume that each level of improvement of airflow circulation and humidity control corresponds to
a \SI{1}{\celsius} drop in the perceived temperature.
The estimated perceived temperatures of each solution are tabulated in \Cref{tbl:comfort}.

\begin{table}[b]
\centering
\begin{tabu}{lrrrr}
    \hline\hline
                            & Solution 1 & Solution 2 & Solution 3 & Solution 4 \\ \hline
Actual temp. (\SI{}{\celsius}) & 28.5    & 28         & 30         & 24         \\
Humidity control            & 2          & 1          & 0          & 3          \\
Airflow circulation         & 3          & 1          & 2          & 1          \\
Perceived temp. (\SI{}{\celsius})& 23.5  & 26         & 28         & 20         \\
    \hline\hline
\end{tabu}
\caption{Expected temperature characteristics}
\label{tbl:comfort}
\end{table}

For \Cref{consider:cost} we consider the cost of each solution over 15 years.
\begin{definition}
    The estimated \textbf{long-term cost} of solution $i$ is
    the sum of its cost of purchasing, installation and the power consumption over 15 years,
    the expected lifespan of this installation.
\end{definition}

The cost is computed by the following formula,
\[ c = \pi + \iota + 33.93 \cdot \omega \]
where \(\pi\) denotes the cost of purchasing, \(\iota\) the cost of installation, and \(\omega\) the power consumption in Watts.
The constant term \(33.93\) reflects the cost of using \(1\) Watt of power over 15 years, extrapolated from current electricity tariffs.\autocite{electricity}

For \Cref{consider:aesthetic} we rate each solution aesthetically out of 5 \davidsstar{}s.
In particular, \Cref{sol1} minimally affects the current aesthetics, while
Solutions \ref{sol2} and \ref{sol3} are considered unsightly.
\Cref{sol4} is largely favourable with a caveat -- the food court will be arbitrarily divided into two halves, both air-conditioned.

The solutions are summarised in \Cref{tab:summary}.

\begin{table}[]
\centering
\begin{tabu}{lrrrr}
    \hline \hline
                         & Solution 1 & Solution 2 & Solution 3 & Solution 4 \\ \hline
    Perceived temp. (\SI{}{\celsius})& 23.5  & 26  & 28         & 20         \\
    Long-term cost (\$)  & 75860      & 57495      & 49206      & 1707500     \\
    Aesthetic appeal & \davidsstar\davidsstar\davidsstar\davidsstar\davidsstar  & \davidsstar\davidsstar & \davidsstar    & \davidsstar\davidsstar\davidsstar\davidsstar     \\
    \hline \hline
\end{tabu}
\caption{Summary of solutions}
\label{tab:summary}
\end{table}

\subsection{Choice of solution}

With reference to \Cref{tab:summary}, it is clear that \textbf{\Cref{sol1}} emerges as the best choice.
Having a long-term cost two orders of magnitude lower than that of \Cref{sol4},
it provides comparable performance, greatly surpassing that of \Cref{sol2} and \Cref{sol3}.
In addition, it is also the most aesthetically appealing.

\section{Evaluation}

\noindent\emph{This section addresses Question 2.}

\subsection{Potential risks}

The solution chosen allows for dust to accumulate and dust mites to gather.
This could pose hygiene and air quality issues for our students and staff, especially those
with respiratory conditions such as asthma.
An unclean fan blows dust and dust mites around the environment, posing as a health risk especially for
people with allergic reactions.\autocite{dustmites}
It is therefore important for the giant ceiling fan to be kept clean in order to
maintain the air quality and the health of our students and staff.

\subsection{Mitigation strategies}

To mitigate this potential risk attributed to \Cref{sol1},
we propose a cleaning schedule for the ceiling fan.\autocite{howtoclean}
\begin{itemize}
    \item At least once every two weeks, the fan is to be dusted clean.
    \item At least once every month, the fan should be brushed and vacumned thoroughly.
\end{itemize}

\section{Future improvements}

For \Cref{consider:comfort}, more advanced models could be tested and experiments conducted in order to determine the precise relationship between a solution's airflow circulation, humidity control and the perceived hotness experienced.

In the determination process of \Cref{consider:cost}, predictions could be made on future electricity tariffs in order to better estimate long-term cost.

In the process of judging \Cref{consider:aesthetic}, the author's personal tastes and preferences were invoked as judgement.
As improvement, we could come up with mock-up designs for each solution,
and allow the students to choose the one they like most.


\printbibliography

\end{document}
