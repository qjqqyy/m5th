\PassOptionsToPackage{unicode=true}{hyperref} % options for packages loaded elsewhere
\PassOptionsToPackage{hyphens}{url}
\PassOptionsToPackage{dvipsnames,svgnames*,x11names*}{xcolor}
%
\documentclass[british,a4paper,]{article}
%\usepackage{setspace}
%\doublespacing
\usepackage[british]{babel}
\title{GEQ1000 Final Reflection Paper}
\author{Qi Ji (A0167793L) [D33]}
\date{19th November 2018}
\makeatletter
\let\thetitle\@title
\let\theauthor\@author
\let\thedate\@date
\makeatother

\usepackage{lmodern}
\usepackage{amssymb,amsmath,amsthm}
\usepackage{ifxetex,ifluatex}
\ifnum 0\ifxetex 1\fi\ifluatex 1\fi=0 % if pdftex
  \usepackage[T1]{fontenc}
  \usepackage[utf8]{inputenc}
  \usepackage{textcomp} % provides euro and other symbols
\else % if luatex or xelatex
  \usepackage{unicode-math}
  \defaultfontfeatures{Ligatures=TeX,Scale=MatchLowercase}
\fi
\usepackage{xcolor}
\usepackage{hyperref}
\usepackage{fancyhdr}
\usepackage{siunitx}
\usepackage{cleveref}
%\usepackage{tabu}
%\usepackage[tablewithin=section]{caption}

\hypersetup{
            pdftitle={GEQ1000 Final Reflection Paper},
            pdfauthor={Qi Ji},
            colorlinks=true,
            linkcolor=Violet,
            citecolor=PineGreen,
            urlcolor=MidnightBlue,
            breaklinks=true}

% use upquote if available, for straight quotes in verbatim environments
\IfFileExists{upquote.sty}{\usepackage{upquote}}{}
% use microtype if available
\IfFileExists{microtype.sty}{%
\usepackage[]{microtype}
\UseMicrotypeSet[protrusion]{basicmath} % disable protrusion for tt fonts
}{}

% set default figure placement to htbp
\makeatletter
\def\fps@figure{htbp}
\makeatother

\usepackage{csquotes}
%% load polyglossia as late as possible as it *could* call bidi if RTL lang (e.g. Hebrew or Arabic)
%\usepackage{polyglossia}
%\setmainlanguage[variant=british]{english}
\usepackage{biblatex}
\addbibresource{\jobname.bib}

\newtheorem{theorem}{Theorem}
\newtheorem{proposition}[theorem]{Proposition}
\newtheorem{lemma}[theorem]{Lemma}
\newtheorem{axiom}[theorem]{Axiom}
\theoremstyle{definition}
\newtheorem{definition}[theorem]{Definition}
\newtheorem{example}[theorem]{Example}
\newtheorem{question}[theorem]{Question}
\theoremstyle{remark}
\newtheorem{remark}[theorem]{Remark}

\newcommand{\set}[1]{\left\{\, #1 \,\right\}}

\pagestyle{fancy}
\lhead{GEQ1000}
\chead{}
\rhead{\thedate}
\lfoot{Qi Ji}
\cfoot{}
\rfoot{\thepage}
%\renewcommand{\footrulewidth}{\headrulewidth}

%--------------------------------------------------------------------------------------------------
\begin{document}
\maketitle

\begin{abstract}
    In this paper I show that questioning is similar across disciplines.
%    Even with my limited knowledge, the links between disciplines are too strong to disregard.
\end{abstract}

At a shallow level, there are dialectic changes depending on the context of the discipline.
What topics in question varies greatly, and so appears the methods of questioning.
We discuss classic questions from physics and economics.

\begin{question}[Quantum Mechanics] \label{particleinabox}
    Given a particle in a box, in what ways can it behave?
\end{question}

\begin{question}[Prisoner's dilemma] \label{prisoner}
    In a game similar to the prisoner's dilemma, how are players likely to behave?
\end{question}

Most would agree that physics and economics are not closely-related disciplines, and
the two questions raised are extensionally very different,
\Cref{particleinabox} asks about the behaviour of physical particles, while
\Cref{prisoner} asks about economic behaviour of people in society.

What we need to expose the similarity of these two questions is a tool mainly used in computer science -- abstraction.
Zoom out far enough and the underlying processes become increasingly similar.
The present way to answer \Cref{particleinabox} is to basically ask,
\begin{quote}
    What are the logical consequences of the Dirac-von Neumann axioms of quantum mechanics\autocite{vonNeumannQM}
    together with a statement saying ``I have a particle in a box''?
\end{quote}
How we deal with \Cref{prisoner} nowadays uses a similar approach,
\begin{quote}
    Consider a suitable formalism of game theory, for example in \autocite{vonNeumannGame},
    express the prisoner's dilemma in said formalism.
    What are the logical consequences?
\end{quote}
These approaches are attributed to polymath John von Neumann, who devised rigorous formalism to both disciplines,
providing concrete logical foundations that enable others to experiment with and apply the model.
Modelling abstracts a problem away, only keeping its logical essence, then by logic, we can
reduce a complicated theory to a simpler set of underlying assumptions.

In the lectures, we have seen how modelling plays a role in physics, computing, engineering and economics.
This is not a coincidence, sensible modelling provides us an indispensable tool for testing our assumptions.
The Wason selection task in philosophy tells us how to proceed from here.
For example in quantum mechanics, we can start from the Dirac-von Neumann axioms and check if particles indeed obey our predictions.
Conversely, we can look for objects whose existence or behaviour contradicts our assumptions.

To me, modelling is the act of reducing a theory to its logical foundations, and its applicability shows the power of raw logic.
This process could be further generalised, according to philosophers Ramsey and Lewis\autocite{lewis1970}.
However, this foundational approach might not always produce intuitive results, which can be illustrated by an example from mathematics.
Starting from a commonly accepted postulate.
\begin{axiom}[Axiom of Choice (rephrased)] \label{AC}
    Given arbitrarily many bags of socks, each non-empty, I can \emph{always}
    choose one sock from each.
\end{axiom}
In the case where I only have finitely many bags of socks, \Cref{AC} is obviously true,
so its full statement is a mere generalisation to arbitrarily many bags, possibly infinite.
However, this innocent generalisation logically entails this counter intuitive result in geometry.
\begin{theorem}[Banach-Tarski paradox \autocite{banachtarski}]
    Take a solid hypothetical ball in 3-dimensional space,
    there is a way to cut the ball into finitely many pieces,
    and reassemble them into two identical copies of the ball you started with.
\end{theorem}

As the examples illustrate, the study of logical consequences is a recurring theme in questioning.
Fortunately, there is an entire field devoted to this -- model theory.
With the power of the axiomatic approach, as exemplified by von Neumann, Hilbert and Ramsey,
model theory provides us powerful tools to reason about truth and consequence.

Again we use the example of quantum mechanics,
physicists have long argued over differing \emph{interpretations} of quantum mechanics, specifically the Dirac-von Neumann axioms.
\begin{definition}
    An \textbf{interpretation} is any abstract structure that fulfills the model we started with.
\end{definition}
\begin{example}[Computational Thinking Lectures]
    A road network tagged with travel times interprets a weighted graph.
\end{example}
Similarly in physics and economics, an interpretation is a way to use the theory to explain reality.
As described in physics videos, scientific theories merely approximate the truth,
even in the purely theoretical context of model theory, a set of statements could be satisfied by many different structures.
The differing interpretations of quantum mechanics all satisfy the basic conditions, but each of them extends it in a different way.

In fact, during the philosophy video on poetry, where Professor Holbo gave an interpretation to his wife's poem,
I was reminded of a well-known result related to model theory.
\begin{theorem}[Completeness Theorem -- Gödel, Henkin] \label{complete}
    In a standard deductive system of logic, any logical consequence could be deduced (via formal proof).
\end{theorem}
One is more technical than the other, but zooming out really far back, the essence of both are the same.
Actually, the proof of \Cref{complete} by Henkin\autocite{henkin} does exactly what Professor Holbo did,
assigning meaning to symbols such that everything turns out to be consistent in the end.

With enough abstraction, the similarity of questioning across seemingly unrelated disciplines is unavoidable.
As long as one is willing to think abstractly and generalise their thoughts to see the common patterns,
learning a few disciplines is enough to unlock the subject of questioning.

\printbibliography
\end{document}
