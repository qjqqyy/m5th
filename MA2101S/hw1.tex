% call TexNewMathZone("D","align*",1)
\documentclass{article}
\usepackage[british]{babel}
\usepackage[a4paper]{geometry}
\usepackage{sectsty}
\parindent 0pt
\allsectionsfont{\normalfont\sffamily\bfseries}

\author{Qi Ji\\\small A0167793L}
\title{MA2101S Homework 1}
\date{31st January 2018}

\usepackage{amsthm, amsmath, amssymb}
\usepackage{enumitem}

%\newtheorem{thm}{Theorem}
%\newtheorem*{proposition}{Proposition}
%\newtheorem{lemma}[thm]{Lemma}
%\newtheorem{corollary}[thm]{Corollary}
\theoremstyle{definition}
\newtheorem{problem}{Problem}
\newtheorem*{definition}{Definition}
\newtheorem*{notation}{Notation}

\numberwithin{equation}{problem}
\newenvironment{solusionssolusionssolution}{
    \renewcommand{\qedsymbol}{$\blacksquare$}
    \begin{proof}[Solution]
    }
    {
    \end{proof}
}

\newcommand{\N}{\mathbb{N}}
\newcommand{\R}{\mathbb{R}}
\newcommand{\Q}{\mathbb{Q}}
\newcommand{\C}{\mathbb{C}}
\newcommand{\set}[1]{\left\{\,#1\,\right\}}
\renewcommand{\vec}[1]{\mathit{#1}_V}
\newcommand{\conj}[1]{\overline{#1}}

\usepackage{fancyhdr}
\pagestyle{fancy}
\lhead{Qi Ji -- A0167793L}
\rhead{MA2101S -- Homework $1$}

\begin{document}
% uncomment if desired
\maketitle

% P 1
\begin{problem}
    Let $\alpha \in \Q$ be a rational number such that the polynomial
    $T^2 - \alpha = 0$ has no solutions in $\Q$.

    Show that $\Q(\sqrt{\alpha}) := \set{ a + b\sqrt{\alpha} \in \C: a,b\in \Q}$,
    where $\sqrt{\alpha}\in\C$ is a square root of $\alpha$,
    is a field under the usual arithmetic operations in $\C$ (is a subfield of $\C$).
\end{problem}
\begin{proof}
    To prove $\Q(\sqrt{\alpha})$ is a subfield of $\C$,
    it suffices to just check for closure under addition, multiplication, negation, reciprociation,
    and the existence of $0$ and $1$.

    \paragraph{Presence of $0$ and $1$.}
    $0,1\in \Q$, and since $0 = 0 + 0\sqrt{\alpha}$ and $1 = 1 + 0\sqrt{\alpha}$, $0,1\in \Q(\sqrt{\alpha})$.

    \paragraph{Closure under $+$.} For any pair $p,q \in \Q(\sqrt{\alpha})$,
    $\exists x, y, w, z\in \Q$ such that
    \begin{align*}
        p &= x + y\sqrt{\alpha}  \\
        q &= w + z\sqrt{\alpha}
    \end{align*}
    Then by associativity of $+$ and distributivity of $\cdot$ over $+$ in $\C$,
    \begin{align*}
        p + q &= x + y\sqrt{\alpha} + w + z\sqrt{\alpha}    \\
        &= (x + w) + (y + z)\sqrt{\alpha}
    \end{align*}
    Because $\Q$ is a field, $x + w$ and $y + z$ are in $\Q$, thus $p+q\in \Q(\sqrt{\alpha})$.

    \paragraph{Closure under $\cdot$.} For any pair $p,q \in \Q(\sqrt{\alpha})$,
    $\exists x, y, w, z\in \Q$ such that
    \begin{align*}
        p &= x + y\sqrt{\alpha}  \\
        q &= w + z\sqrt{\alpha}
    \end{align*}
    Then by associativity of $\cdot$ and distributivity,
    \begin{align*}
        p\cdot q &= (x + y\sqrt{\alpha}) \cdot (w + z\sqrt{\alpha}) \\
        &= x(w + z\sqrt{\alpha}) + y\sqrt{\alpha}(w + z\sqrt{\alpha})   \\
        &= x w + x z \sqrt{\alpha} + y w \sqrt{\alpha} + y z \sqrt{\alpha} \sqrt{\alpha}    \\
        &= (x w + y z \alpha) + (x z + y w) \sqrt{\alpha}
    \end{align*}
    Again because $\Q$ is a field, $x w + y z \alpha$ and $x z + y w$ are in $\Q$,
    thus $p\cdot q\in \Q(\sqrt{\alpha})$.

    \paragraph{Closure under $-$.} For any $p \in \Q(\sqrt{\alpha})$,
    $\exists x, y \in \Q$ such that $p = x + y\sqrt{\alpha}$.
    Then
    \[-p = -(x + y\sqrt{\alpha}) = -x + (-y)\sqrt{\alpha}.\]
    $-p \in \Q(\sqrt{\alpha})$ because $-x, -y$ in $\Q$ due to $\Q$ being a field.

    \paragraph{Closure under $(-)^{-1}$.} For any $p \in \Q(\sqrt{\alpha})\setminus \set{0}$,
    $\exists x, y \in \Q$ such that $p = x + y\sqrt{\alpha}$.
    Then compute $p^{-1}$ in $\C$ as follows,
    \begin{align*}
        p^{-1} &= \frac{1}{p} = \frac{1}{x + y\sqrt{\alpha}}    \\
        &= \frac{x - y\sqrt{\alpha}}{x^2 - y^2 \alpha}
    \end{align*}
    \textbf{Claim.} $x^2 - y^2 \alpha \ne 0$.\\
    Case $y=0$, then because $p\ne 0, x\ne 0$, so $x^2 - y^2 \alpha \ne 0$.
    Case $y\ne 0$, then $\exists y^{-1} \in \Q$. Suppose for a contradiction $x^2 - y^2\alpha = 0$, then we have
    \begin{align*}
        (y^{-1})^2 (x^2 - y^2\alpha) &= 0 \\
        \left(\frac{x}{y}\right)^2 - \alpha &= 0
    \end{align*}
    but since $\Q$ is a field and $y \ne 0$ by assumption, $(\frac{x}{y})^2 \in \Q$,
    contradicting with fact that $T^2 - \alpha = 0$ has no solution in $\Q$.
    Hence $x^2 - y^2 \alpha \ne 0$ and its reciprocal exists in $\Q$.

    Therefore $p^{-1} = \dfrac{x}{x^2 - y^2 \alpha} - \dfrac{y}{x^2 - y^2 \alpha} \sqrt{\alpha}$,
    and because $x,y,\alpha \in \Q$, $p^{-1} \in \Q(\sqrt{\alpha})$.
\end{proof}

% P 2
\begin{problem}
    Define $V := \C^\C$, consider the following subsets of $V$.
    Which are $\R$-vector spaces? Which are $\C$-vector spaces? Justify.
\end{problem}

\begin{notation}
    Let $\vec{0} : \C \to \C, z \mapsto 0$ denote the zero vector which is the constant function of $0_\C$.
    For each part, let the subset be called $W$.
\end{notation}
Due to the sets below being subsets of $V$ under the same operations,
it suffices to check if $W$ in each part is a subspace
having $\vec{0}$, closure under addition and scalar multiplication.

\begin{enumerate}[label=(\roman*)]
% part i
    \item all $f\in V$ such that $f(0) = 1$;

        Because $\vec{0}(0) = 0 \ne 1$, $W$ is not a vector space due to absense of $\vec{0}$.
        \hfill$\blacksquare$
% part ii
    \item all $f\in V$ such that $f(0) = f(1)$;

        $\vec{0}(0) = \vec{0}(1) = 0$, so the zero vector is in $W$.

        Take any pair $f,g \in W$, then
        \begin{align*}
            (f+g)(0) &= f(0) + g(0) \\
            &= f(1) + g(1)  \\
            &= (f+g)(1)
        \end{align*}
        closure under vector addition holds.

        Take any $f \in W$, then for any $k\in\C$ and $k \in\R$,
        \begin{align*}
            (kf)(0) &= k\cdot f(0)  \\
            &= k\cdot f(1)  \\
            &= (kf)(1)
        \end{align*}
        $W$ is both a $\R$ and $\C$-vector space.
        \hfill$\blacksquare$
% part iii
    \item all $f\in V$ such that for every $z\in \C$, one has $\conj{f(z)} = f(z)$;

        $\forall z\in \C.~ \conj{\vec{0}(z)} = \conj{0} = 0$, so $\vec{0}$ is in $W$.

        For any $z\in \C$, $\conj{z} = z \iff z \in \R$.\\
        Take any pair $f,g \in W$,
        then for any $z\in\C, (f+g)(z) = f(z) + g(z)$,
        since $f(z), g(z) \in \R$ and $\R$ is a subfield of $\C$,
        $(f+g)(z) \in \R$ and thus closure under vector addition holds.

        Take any non-zero $f$ from $W$, take any $k\in \C$ where $k = a + bi, a,b\in \R, b\ne 0$,
        then for any $z\in \C$ where $f(z) \ne 0$,
        \begin{align*}
            (kf)(z) &= k\cdot f(z) \\
            &= (a+bi)\cdot f(z) \\
            &= a\cdot f(z) + (b\cdot f(z))i
        \end{align*}
        Since $\exists k\in \C, f\in W$ where $\operatorname{Im}((kf)(z)) \ne 0 \iff (kf)(z)\not\in \R$,
        closure under scalar multiplication is broken and this set is not a $\C$-vector space.

        However, for any $f\in V$ where for every $z\in \C$, $f(z) \in \R$. For any $k\in \R, z\in \C$,
        because $\R$ is a subfield, $(kf)(z) = k\cdot f(z) \in \R$. Hence $W$ is a $\R$-vector space.
        \hfill$\blacksquare$
% part iv
    \item all $f\in V$ such that for every $z\in \C$, one has $f(\conj{z}) = f(z)$;\\
        $\vec{0}$ is a constant function and ignores its parameter, satisfying the condition, thus $\vec{0} \in W$.

        For any pair $f,g \in W$, then for any $z\in \C$,
        \begin{align*}
            (f+g)(\conj{z}) &= f(\conj{z}) + g(\conj{z}) \\
            &= f(z) + g(z) \\
            &= (f+g)(z)
        \end{align*}
        Thus closure under vector addition holds.

        For any $f \in W, k\in \C$, for all $z\in \C$,
        \begin{align*}
            (kf)(\conj{z}) &= k\cdot f(\conj{z})    \\
            &= k\cdot f(z)  \\
            &= (kf)(z)
        \end{align*}
        Closure under scalar multiplication holds (also holds for $k\in \R$).
        $W$ is both a $\R$ and $\C$-vector space.
        \hfill$\blacksquare$
% part v
    \item all $f\in V$ such that for every $z\in \C$, one has $f(z^2) = f(z)^2$;\\
        Take $f = g = \text{id}_\C$, clear that property above holds.
        Take $z = 2 \in \C$,
        \begin{align*}
            (f+g)(2^2) &= f(4) + g(4)   \\
            &= 8    \\
            (f+g)(2)^2 &= (f(2) + g(2))^2   \\
            &= 4^2 = 16
        \end{align*}
        We can see that $(f+g)(2^2) \ne (f+g)(2)^2$,
        thus $f + g \notin W$,
        closure under vector addition is broken and $W$ is not a $\R$ or $\C$-vector space.
        \hfill$\blacksquare$
\end{enumerate}

\begin{definition}
    Let $K$ be a field, $V$ a $K$-vector space. For any $K$-subspaces $U, W \subseteq V$, define
    \[
        U + W := \set{ v\in V: \exists u\in U, w\in W.~ v = u + w }.
    \]
\end{definition}
% P 3
\begin{problem}
    Let $V:=\R^\R$. Consider $V_\text{even}$ (resp. $V_\text{odd}$) as subsets of all even (resp. odd) functions.
\end{problem}
\begin{notation}
    Let $\vec{0} : \R \to \R, x \mapsto 0$ denote the zero vector which is the constant function of $0_\R$.
\end{notation}
\begin{enumerate}[label=(\alph*)]
    \item Show that $V_\text{even}$ and $V_\text{odd}$ are $\R$-subspaces of $V$.
        \begin{proof}
            Firstly, $\forall x \in \R.~ \vec{0}(x) = 0$, trivially $\vec{0} \in V_\text{odd}$ and $\vec{0} \in V_\text{even}$.

            Consider $f,g\in V_\text{even}$, then $\forall x\in \R$,
            \begin{align*}
                (f+g)(-x) &= f(-x) + g(-x)  \\
                &= f(x) + g(x) \\
                &= (f+g)(x)
            \end{align*}
            Thus $f+g \in V_\text{even}$.
            Consider any $f\in V_\text{even}, k\in \R$, then $\forall x\in \R$,
            \begin{align*}
                (kf)(-x) &= k\cdot f(-x)    \\
                &= k\cdot f(x)  \\
                &= (kf)(x)
            \end{align*}
            Thus $kf \in V_\text{even}$.
            Therefore $V_\text{even}$ is a subspace of $V$.

            Now consider $f,g \in V_\text{odd}$, then $\forall x\in \R$,
            \begin{align*}
                (f+g)(-x) &= f(-x) + g(-x)  \\
                &= -f(x) -g(x)      \\
                &= -(f(x) + g(x))    \\
                &= -(f+g)(x)
            \end{align*}
            Thus $f+g \in V_\text{odd}$.
            Consider any $f\in V_\text{odd}, k\in \R$, then $\forall x\in \R$,
            \begin{align*}
                (kf)(-x) &= k\cdot f(-x)    \\
                &= k\cdot (-f(x))   \\
                &= -k\cdot f(x) \\
                &= -(k\cdot f(x)) = -(kf)(x)
            \end{align*}
            Thus $kf \in V_\text{odd}$.
            Therefore $V_\text{odd}$ is a subspace of $V$.
        \end{proof}
    \item Show that $V_\text{even} \cap V_\text{odd} = \set{\vec{0}}$ and $V_\text{even} + V_\text{odd} = V$.
        \begin{proof}
            Take $f\in V_\text{even} \cap V_\text{odd}$, $f$ is both even and odd, so $\forall x\in \R$,
            $$f(-x) = f(x) \text{ and } f(-x) = -f(x).$$
            So $f(x) = -f(x)$ for all $x\in \R$, which is the case if and only if for all $x\in \R$, $f(x) = 0$.
            This means $f$ is the constant function of $0$, that is $\vec{0}$, this implies $V_\text{even} \cap V_\text{odd} = \set{\vec{0}}$.

            Consider any $h \in V_\text{even} + V_\text{odd}$, by definition, $\exists f \in V_\text{even}, g \in V_\text{odd}.~ h = f + g$.
            Since $V_\text{even}$ and $V_\text{odd}$ are subspaces of $V$, $h = f + g \in V$,
            so $V_\text{even} + V_\text{odd} \subseteq V$.

            Conversely take $f\in V$. For all $x\in \R$,
            \begin{align*}
                f(x) &= f(x) + \vec{0}(x) \\
                &= \frac{1}{2} f(x) + \frac{1}{2} f(x) + \frac{1}{2} f(-x) - \frac{1}{2} f(-x)\\
                &= \frac{1}{2} (f(x) + f(-x)) + \frac{1}{2} (f(x) - f(-x))
            \end{align*}
            Define $g,h : \R \to \R$ as
            \begin{align*}
                g: x &\mapsto \frac{1}{2} (f(x) + f(-x)) \\
                h: x &\mapsto \frac{1}{2}  (f(x) - f(-x))
            \end{align*}
            Verify that $g\in V_\text{even}$ and $h\in V_\text{odd}$.
            \begin{align*}
                g(-x) &= \frac{1}{2}(f(-x) + f(-(-x)))  \\
                &= \frac{1}{2}(f(x) + f(-x))    \\
                h(-x) &= \frac{1}{2}(f(-x) - f(-(-x)))  \\
                &= -\frac{1}{2}(f(x) - f(-x))
            \end{align*}
            Because $f = g + h$, $f \in V_\text{even} + V_\text{odd}$.
            So $V \subseteq V_\text{even} + V_\text{odd}$ and this completes the proof that $V_\text{even} + V_\text{odd} = V$.
        \end{proof}
\end{enumerate}

% P 4
\begin{problem}
    Let $K$ be any field, let $V$ be a $K$-vector space,
    and let $V_1, V_2\subseteq V$ be $K$-subspaces of $V$.
    Suppose $V_1 \cap V_2 = \set{\vec{0}}$ and $V_1 + V_2 = V$.
    Show that for any $v \in V$, there exist unique vectors
    $v_1 \in V_1$ and $v_2 \in V_2$ such that $v = v_1 + v_2$ in V.
\end{problem}
\begin{proof}
    Take any arbitary $v\in V$, since $V_1 + V_2 = V$,
    by definition of $V_1 + V_2$, there exists $v_1 \in V_1, v_2 \in V_2$ such that $v = v_1 + v_2$.

    Suppose $\exists v_1' \in V_1, v_2' \in V_2$ where $v = v_1' + v_2'$.
    \begin{align*}
        v = v_1 + v_2 &= v_1' + v_2'    \\
        v_1 - v_1' &= v_2' - v_2
    \end{align*}
    Clearly $LHS \in V_1$ and $RHS \in V_2$ due to closure under vector addition.
    This implies $LHS = RHS = \vec{0}$ since $V_1 \cap V_2 = \set{\vec{0}}$, thus we have
    $v_1 = v_1'$ and $v_2 = v_2'$, completing the uniqueness proof.
\end{proof}

% P 5
\begin{problem}
    Let $K$ be any field, let $V$ be a $K$-vector space,
    and let $V_1, V_2\subseteq V$ be $K$-subspaces of $V$.
    Suppose the set-theoretic union $V_1 \cup V_2$ is also a $K$-subspace of $V$.
    Show that one of the subspaces $V_1$ or $V_2$ is contained in the other.
\end{problem}
\begin{proof}
    $V_1\cup V_2$ is also a $K$-subspace of $V$,
    Suppose for a contradiction neither $V_1$ nor $V_2$ is contained in the other.
    Means that $V_1 \setminus (V_1 \cap V_2)$ and $V_2 \setminus (V_1 \cap V_2)$ are both non-empty,
    namely there exists $v_1 \in V_1, v_2 \in V_2$ where $v_1, v_2\not \in V_1 \cap V_2$.
    Clearly $v_1, v_2 \in V_1 \cup V_2$.
    By closure property of vector addition in a subspace, $v_1 + v_2 \in V_1 \cup V_2$.
    Case $v_1 + v_2 \in V_1$, then $v_1 + v_2 - v_1 = v_2 \in V_1$, contradicting fact that $v_2 \not \in V_1\cap V_2$.
    A symmetric argument shows that $v_1 + v_2$ cannot be in $V_2$, thus a contradiction.
\end{proof}

% P 6
\begin{problem}
    Let $K$ be an infinite field, let $V$ be a vector space over $K$,
    and let $V_1, \dots V_n \subset V$ be a finite list of proper $K$-subspaces over V.
    Show that $V \neq \displaystyle\bigcup_{j=1}^n V_j$.
\end{problem}
\begin{proof}
    Let $n\in \N$, $V_1, \dots, V_n \subset V$ be a finite list of proper $K$-subspaces over $V$.
    Suppose for a contradiction that $V = \displaystyle\bigcup_{i=1}^n V_i$.
    Trivially, $n$ cannot be $0$ or $1$, result of Q5 implies $n \ne 2$, so $n \ge 3$.

    Using an algorithm, we can remove subspaces in the list as such,
    \begin{enumerate}
        \item For each $x \in \set{1,2,\dots,n}$,
        \item If $\displaystyle\bigcup_{i\ne x} V_i = V$,
            remove $V_x$ from the list.
    \end{enumerate}
    Since list is finite, algorithm halts.
    Therefore, without loss of generality, we can assume that for any $x \in \set{1,\dots,n}$,
    \begin{align*}
        V = \bigcup_{i=1}^n V_i \ne &\bigcup_{i\ne x} V_i, \text{ and}   \\
        V_x \setminus &\bigcup_{i \ne x} V_i \ne \emptyset.
    \end{align*}

    Now take vectors
    $$ u \in V_1 \setminus \bigcup_{i\ne 1} V_i \text{ and } w \in V_2\setminus \bigcup_{i\ne 2} V_i $$
    For any $a\in K \setminus \set{0_K}$, define
    $$v_a := u + aw.$$
    It is clear that $v_a \in V = \displaystyle\bigcup_{i=1}^n V_i$.
%
    Suppose $v_a \in V_1$, then $a^{-1}(v_a - u) = w \in V_1$, contradicting $w \notin V_i$ for any $i \ne 2$.
    So $v_a \notin V_1$.
    Suppose $v_a \in V_2$, then $v_a - aw = u \in V_2$, contradicting $u \not\in V_i$ for any $i \ne 1$.
    So $v_a \in \displaystyle\bigcup_{i=3}^n V_i$.

    $|K \setminus\set{0_K} | = \infty$ while $|\set{V_3, \dots, V_n}| = n- 2$.
    Since $|K \setminus\set{0_K} | > |\set{V_3, \dots, V_n}|$, there does not exist an injective map
    $K \setminus\set{0_K} \to \set{V_3, \dots, V_n}$.
    Now consider the following maps,
    \begin{align*}
        f: K\setminus\set{0_K} &\to \bigcup_{i=3}^n V_i    \\
            a &\mapsto v_a = u + aw; \\
        g: \bigcup_{i=3}^n V_i &\to \set{V_3, \dots, V_n}   \\
            v &\mapsto V_j
    \end{align*}
    where $V_j$ is the lowest-indexed subspace fulfilling $v \in V_j$.

    We have $g \circ f: K \setminus\set{0_K} \to \set{V_3, \dots, V_n}$, which as shown, cannot be injective.
    This means that $\exists a,b \in K \setminus\set{0_K}, V_j \in \set{V_3, \dots, V_n}.~ a \ne b \land (g\circ f)(a) = (g\circ f)(b)$. So we can conclude that 
%    Consider all vectors of the form $v_a = u + aw$,
%    $$\forall a\in K\setminus\set{0_K}, \exists j\in \set{3,4,\dots,n}.~  v_a \in V_j.$$
%    Since there are infinitely many distinct values of $a$ but only finitely many subspaces for $v_a$ to be in,
%    by a cardinality argument we can conclude that
    $$\exists V_j\in\set{V_3, \dots, V_n},~ a,b\in K\setminus\set{0_K}.~ a\ne b \text{ and } v_a, v_b\in V_j.$$

    Then $v_a - v_b \in V_j$ due to closure property of subspace,
    \begin{align*}
        v_a - v_b &= (u + aw) - (u + bw)    \\
        &= (a-b)w
    \end{align*}
    Since $a\ne b$, $a-b \ne 0_K$, by closure property this implies $(a-b)^{-1} \cdot (a-b)w \in V_j \implies w \in V_j$,
    contradicting fact that $\forall i \ne 2.~ w \notin V_i $.

    Therefore $V$ cannot be a union of a finite list of proper $K$-subspaces.
\end{proof}

\end{document}
